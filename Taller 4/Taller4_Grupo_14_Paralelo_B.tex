%%%%%%%%%%%%%%%%%%%%%%%%%%%%%%%%%%%%%%%%%%%%%%%%%%%%%%%%%%%%%%%%%%%%%%%%%%%
%
% Plantilla para un art�culo en LaTeX en espa�ol.
%
%%%%%%%%%%%%%%%%%%%%%%%%%%%%%%%%%%%%%%%%%%%%%%%%%%%%%%%%%%%%%%%%%%%%%%%%%%%

\documentclass[12pt]{article}

% Esto es para poder escribir acentos directamente:
\usepackage[latin1]{inputenc}
% Esto es para que el LaTeX sepa que el texto est� en espa�ol:
\usepackage[spanish]{babel}

% Paquetes de la AMS:
\usepackage{amsmath, amsthm, amsfonts}

% Teoremas
%--------------------------------------------------------------------------
\newtheorem{thm}{Teorema}
\newtheorem{cor}[thm]{Corolario}
\newtheorem{lem}{Lema}
\newtheorem{prop}[thm]{Proposici�n}
\theoremstyle{definition}
\newtheorem{defn}{Definici�n} %{defn}{thm} se utiliza para enumerar tambien el teorema sin escribir el {thm} solo sale el numero de definicion
\theoremstyle{remark}
\newtheorem{rem}[thm]{Observaci�n}

% Atajos.
% Se pueden definir comandos nuevos para acortar cosas que se usan
% frecuentemente. Como ejemplo, aqu� se definen la R y la Z dobles que
% suelen representar a los conjuntos de n�meros reales y enteros.
%--------------------------------------------------------------------------

\def\RR{\mathbb{R}}
\def\ZZ{\mathbb{Z}}

% De la misma forma se pueden definir comandos con argumentos. Por
% ejemplo, aqu� definimos un comando para escribir el valor absoluto
% de algo m�s f�cilmente.
%--------------------------------------------------------------------------
\newcommand{\abs}[1]{\left\vert#1\right\vert}

% Operadores.
% Los operadores nuevos deben definirse como tales para que aparezcan
% correctamente. Como ejemplo definimos en jacobiano:
\Large
\usepackage{anysize}
\papersize{29.7cm}{21.0cm} %para taman?o carta, para otros eligieran la correcta 
\marginsize{3.0cm}{3.0cm}{2.5cm}{2.5cm}
%\marginsize{Izque}{Derec}{Arrib}{Abajo}
%--------------------------------------------------------------------------
\DeclareMathOperator{\Jac}{Jac}

%--------------------------------------------------------------------------
\title{INTERPOLACION POLINOMIAL}

\author{V�ctor Torres, Juan Benalcazar}

\date{Junio 01 de 2016}

%\date{\today}
\begin{document}
\maketitle

Este es mi primer documento en \LaTeX. Comenzamos escribiendo un peque�o  p�rrafo que presenta una clase muy importante de sistemas lineales:

``La soluci�n num�rica de sistemas lineales de punto de ensilladura constituye un t�pico muy importante en la formulaci�n y desarrollo de una gran cantidad de problemas de las ciencias computacionales y la ingenier�a. Las dimensiones de este tipo de sistemas as� como su patr�n de dispersi�n son variados y dependen del tipo de aplicaci�n involucrada, pero en general poseen una estructura en bloques de la forma:

%\abstract{Completar!}

%\section{Secci�n}
%Completar! \cite{Cd94}

\begin{equation}
\underbrace{
\begin{bmatrix}
A & B \\ 
A^{T} & O
\end{bmatrix}
}_{\mathcal{A}}
\underbrace{
\begin{bmatrix}
x\\
\lambda
\end{bmatrix}
}_{u}
=
\underbrace{
\begin{bmatrix}
f\\
g
\end{bmatrix}
}_{b}
\end{equation}




donde $A \in \mathbb{R}^{n\times n}, B  \in \mathbb{R}^{m\times n}, O \in \mathbb{R}^{m\times m} $ es una matriz nula, $f\in \mathbb{R}^{n}$ , 
$g\in \mathbb{R}^{m}$ y $n \geq m".$

Para practicar la escritura de normas, ra�ces cuadradas, limites y series, considere el siguiente fragmento de texto en donde se define la exponencial de un operador lineal $T : \mathbb{R}^{^{n}} \rightarrow \mathbb{R}^{n}$

Para definir la exponencial de un operador lineal $T : \mathbb{R}^{n} \rightarrow \mathbb{R}^{n}$ es necesario establecer el concepto de convergencia  en el espacio lineal  $\mathcal{L}(\mathbb{R}^{^{n}})$ de los operadores lineales sobre $\mathbb{R}^{n}$. Para ello, se define el $operador$ $norma$ $de$ $T$ como:

\begin{equation}
\left \| T \right \| = \max_{\left | x \right |\leq 1}\left | T(x) \right |,
\end{equation}
donde $\left| x \right |$ denota la forma Euclidea de $x :=(x_1,x_2,...,x_n) \in \mathbb{R}^{n},$ esto es, 
\begin{equation}
\left| x \right | = \sqrt{x_1 + x_2 + ...+ x_n.}
\end{equation}

La norma de operadores posee las propiedades usuales de una norma, por lo tanto, si $S, T \in \mathcal{L}(\mathbb{R}^{n}):$
\begin{itemize}
\item $\left \| T \right \|\geq 0 $ y $\left \| T \right \|=0$ si y solo si $T=0$
\item $\left \| T \right \| = \left | k \right |\left \| T \right \|$ para todo $k \in \mathbb{R}.$
\item$\left \| S + T\right\|\leq \left \|S\right\| + \left \| T\right\|.$
\end{itemize}
\begin{defn}
$ Una$  $sucesi\acute{o}n$ $de$ $operadores$ $lineales$ $T_k \in \mathcal{L}(\mathbb{R}^{n})$ $converge$ $a$ $un$ $operador$ $lineal$  $T \in \mathcal{L}(\mathbb{R}^{n})$  $cuando$ $k\rightarrow\infty$ $si:$ 
\begin{equation}
 \lim_{k\rightarrow\infty} T_k = T
\end{equation}
$ esto$ $es$, $para$ $todo$ $\varepsilon > 0$ $existe$ $N \in \mathbb{N}$ $ tal$ $que$ $si$ $k \geq N$ $entonces$ $\left \|T - T_k\right\| < \varepsilon$.
\end{defn}
\begin{lem}
$ Para$ $S,$ $T \in \mathcal{L}(\mathbb{R}^{n})$ $y$ $ x \in \mathbb{R}^{n} ,$
\begin{itemize}
\item $\left \|T(x) \right \|\leq \left \|T \right\| \left \|x\right\|$.
\item $\left\|TS \right \|\leq \left\|T\right\| \left\| S\right\|$.
\item $\left\| T^{k} \right\|\leq \left\|T\right\|^{k}$ $para$ $k$ $= 0,1,2,...$
\end{itemize}
\end{lem}
\begin{thm}
Dado $T \in \mathcal{L}(\mathbb{R}^{n})$ y $t_0 > 0.$ la serie
\begin{equation}
\sum_{k=0}^{\infty}\frac{T^{k}t^{k}}{k!},
\end{equation}
es absolutamente y uniformemente convergente para todo $\left |t\right| \leq t_0.$
\end{thm}


En virtud del teorema 1 se define la exponencial de un operador $T$ mediante la serie absolutamente convergente:
\begin{equation}
e^{^{T}} := \sum_{k=0}^{\infty}\frac{T^{k}}{k!} . 
\end{equation}

 Para practicar la escritura de derivadas, presentamos la ecuaci�n de Euler y la metodolog�a para transformarla en una ecuaci�n de segundo orden con coeficientes constantes, as� como un sistema de ecuaciones diferenciales. 
La ecuaci�n de Euler $x^{2}y^{\prime\prime} + \alpha x y^{\prime} + \beta y = 0$ se puede reducir a una ecuaci�n con coeficientes constantes mediante un cambio de la variable independiente.
Sea $x = e^{z}$ o $z= ln(x)$, y considere solo el intervalo $ x>0$, entonces:
\begin{equation}
\frac{dy}{dx} = \frac{1}{x} \frac{dy}{dz}
\end{equation}
\begin{equation}
\frac{d^{2}y}{dx^{2}} = \frac{1}{x^{2}} \frac{d^{2}y}{dz^{2}} - \frac{1}{x^{2}} \frac{dy}{dz}
\end{equation}
y en consecuencia la ecuacion de Euler se convierte en la siguiente:
\begin{equation}
\frac{d^{2}y}{dz^{2}} + (\alpha - 1) \frac{dy}{dz} + \beta y = 0
\end{equation}
Un ejemplo de un sistema de ecuaciones diferenciales no homogeneo es el siguiente:
\begin{equation}
\begin{aligned}
\dot{x} &= -2x + y + 2 e^{-t} \\ \dot{y} &= x - 2y + 3t,
\end{aligned}
\end{equation} %IMPORTANTE ESCRIBIR EL & para que las ecuaciones se centren.












%\subsection{Subsection}\label{sec:nada}
%Completar!

%\subsubsection{Subsubsection}\label{sec:nada2}
%Completar!

% Bibliograf�a.
%-----------------------------------------------------------------
%\begin{thebibliography}{99}

%\bibitem{Cd94} Autor, \emph{T�tulo}, Revista/Editor, (a�o)

%\end{thebibliography}

\end{document} 